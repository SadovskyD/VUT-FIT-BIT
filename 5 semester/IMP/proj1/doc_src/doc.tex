\documentclass[11pt,a4paper]{article}

\usepackage[left=2cm,text={17cm,24cm},top=3cm]{geometry}
\usepackage[bookmarksopen,colorlinks,plainpages=false,urlcolor=blue,unicode,linkcolor=blue]{hyperref}
\usepackage[czech]{babel}
\usepackage[utf8]{inputenc}
\usepackage[T1]{fontenc}
\usepackage{indentfirst}
\usepackage{graphicx}
\usepackage{enumitem}
\usepackage{float}
\usepackage{url}

\graphicspath{{.}}

\begin{document}

% #################################################################################################
% TITLEPAGE

\begin{titlepage}
    \begin{center}
        \Huge
        \textsc{
            Fakulta informačních technologií\\
            Vysoké učení technické v~Brně
        }
        \vspace{100px}
        \begin{figure}[!h]
            \centering
            \includegraphics[scale=0.3]{img/vutbr-fit-logo.eps}
        \end{figure}
        \\[20mm]
        \huge{
            \textbf{
                IMP - Mikroprocesorové a vstavané systémy
            }
        }
        \\[2em]
        \LARGE{
            \textbf{
                Sada demo aplikací nad vestavným OS
            }
        }
        \vfill
    \end{center}
        \Large{
            Dmytro Sadovskyi (xsadov06) \hfill \today
        }

\end{titlepage}

% #################################################################################################
% CONTENT

\setlength{\parskip}{0pt}
\hypersetup{hidelinks}\tableofcontents
\setlength{\parskip}{0pt}

\newpage %#########################################################################################

\section{Úvod}

    Zadáním projektu bylo se seznámit s principem tvorby vestavných aplikaci na vybraném technickém vybavení
a vytvořit dva projekty
    \begin{itemize}
        \item Jednoduchý (bare-metal) projekt umožňující a demonstrující řízení základních periférií.
        \item Projekt demonstrující řízení základních periférií, dostupných na daném technickém vybavení, z úrovně úloh OS. Pro obsluhu každé z periférií je nutno využít samostatnou úlohu a v řešení využít komunikační popř. synchronizační službu OS a alespoň jednu službu OS pro práci s časem
    \end{itemize} 


\section{Použité prostředky}
    
    \subsection{Hardware}

        \indent Pro tento projekt stačí pouze deska FITkit3 a její periferie. FITkit 3 obsahuje                      mikrokontroler MCU Freescale KINETIS MK60DN512VMD10.

    \subsection{Software}

        \indent Pro implementaci bylo použito vývojové prostředí Kinetis Design Studio verzi 3.2.0.
                Toto prostředí je z technického hlediska nadstavbou prostředí zvaného Eclipse. Součástí tohoto IDE je nástroj zvaný Processor Expert, pomocí kterého je umožněno rychlé a snadné nastavení jádra MCU i jednotlivých periferií, se kterými vestavná aplikace pracuje. Z tohoto důvodu pro implementaci byl Processor Expert použít.

\section{Implementace}
    
    Implementace projektu se nachází v souboru main.c ve složce Sources. Ostatní soubory jsou generovány při vytvoření projektu pomocí KDS další kód generuje Processor Expert.

    \subsection{Bare-metal projekt}
    V rámci "bare-metal" projektu je obsluha demonstrována pomocí LED9, LED12 a tlačítka SW4. Tento projekt se nachází ve složce xsadov06\_baremetal.\\

    V bare-metal projektu jsou využity následující komponenty Processor Expertů:
    \begin{itemize}
        \item Bit1 – BitIO – komponenta, která představuje GPIO pin 5 portu B mikrokontroleru, který je            připojen na LED diodu. Nastaven jako výstupní, počáteční hodnota byla nakofigurovana na log. 1.
        \item Bit2 – BitIO – komponenta, která představuje GPIO pin 27 portu E mikrokontroleru, který je           připojen na tlačítko. Nastaven jako vstupni.
        \item Bit3 – BitIO – komponenta, která představuje GPIO pin 2 portu B mikrokontroleru, který je            připojen na LED diodu. Nastaven jako výstupní, počáteční hodnota byla nakofigurovana na log. 1.
        \item PORTE – komponenta, která představuje port E mikrokontroleru.
    \end{itemize}
    Program vestavné aplikace běží v nekonečné smyčce, a funguje tim zpusobem:        
        \begin{itemize}
        \item Máme proměnnou led, která je zodpovědná za výběr LED diody, pokud hodnota proměnné je 12 bliká       LED12, pokud hodnota proměnné je 9 bliká LED9.
        \item Na začátku hodnota proměnné je 12.
        \item Stisknutím tlačítka SW4 nastavujeme měnime hodnotu proměnné led.
    \end{itemize}
    Pro nastavení rychlosti blikání byla vytvořena funkce pause.
    
    \subsection{FreeRTOS projekt}
    V rámci FreeRTOS projektu je obsluha demonstrována stejným způsobem. Tento projekt se nachází ve složce xsadov06\_FreeRTOS.\\

    V FreeRTOS projektu jsou využity následující komponenty Processor Expertu:
    \begin{itemize}
        \item Bit1 – BitIO – komponenta, která představuje GPIO pin 5 portu B mikrokontroleru, který je            připojen na LED diodu. Nastaven jako výstupní, počáteční hodnota byla nakofigurovana na log. 1.
        \item Bit2 – BitIO – komponenta, která představuje GPIO pin 27 portu E mikrokontroleru, který je           připojen na tlačítko. Nastaven jako vstupni.
        \item Bit3 – BitIO – komponenta, která představuje GPIO pin 2 portu B mikrokontroleru, který je            připojen na LED diodu. Nastaven jako výstupní, počáteční hodnota byla nakofigurovana na log. 1.
        \item PORTE – komponenta, která představuje port E mikrokontroleru.
        \item FRTOS1 – FreeRTOS – komponenta, která představuje port operačního systému FreeRTOS
    \end{itemize}
    Máme globální proměnnou led, která je zodpovědná za výběr LED diody, a na začátku má hodnotu 12.
    Pro obsluhu každé periferie je vytvořené samostatné úlohy běžící v nekonečných smyčcích. Pro obsluhu LED diody LED9 je vytvořena funkce ledD9\_task, pokud hodnota proměnné led je 9 bliká LED dioda 9. Pro obsluhu tlačítka SW4 je vytvořena funkce SW4\_task, pokud tlačítko je stisknuté, hodnota led se mění. Obsluha LED12 funguje stejnm způsobem. Při osluze LED diod byl použit semafor k zabránění blikání obou LED diod současně.
    Pro blikání diody se používá funkce FRTOS1\_vTaskDelay().

\newpage %#########################################################################################

\makeatletter
\makeatother
\bibliographystyle{czechiso}
\nocite{*}
\begin{flushleft}
    \bibliography{quotation}
\end{flushleft}

\end{document} %###################################################################################
